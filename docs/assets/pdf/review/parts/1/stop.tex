\chapterimage{chapter_head_2.pdf}

\chapter{What is Good-Enough?}

Code reviewers should assume that \textbf{good code is not about code itself. It's about software users}.

\section{Keep Your Users in the Trade-Off}

Normally you’re writing software for other people. Often you’ll remember to get requirements from them.2 But how often do you ask them how good they want their software to be? Sometimes there’ll be no choice. If you’re working on pacemakers, the space shuttle, or a lowlevel library that will be widely disseminated, the requirements will be more stringent and your options more limited. However, if you’re working on a brand new product, you’ll have different constraints. The marketing people will have promises to keep, the eventual end users may have made plans based on a delivery schedule, and your company will certainly have cash-flow constraints. It would be unprofessional to ignore these users’ requirements simply to add new features to the program, or to polish up the code just one more time. We’re not advocating panic: it is equally unprofessional to promise impossible time scales and to cut basic engineering corners to meet a deadline.

The scope and quality of the system produced should be specified as part of that system’s requirements.

\begin{marker}
    Make Quality a Requirements Issue
\end{marker}

Often you’ll be in situations where trade-offs are involved. Surprisingly, many users would rather use software with some rough edges today than wait a year for the multimedia version. Many IT departments with tight budgets would agree. Great software today is often preferable to perfect software tomorrow. If you give your users something to play with early, their feedback will often lead you to a better eventual solution

\begin{marker}
    Make sure PR size is not too big. Break big ones to make sure small and good pieces goes in so that users can play early
\end{marker}