\chapterimage{chapter_head_2.pdf}

\chapter{Functions}

Every system is built from a domain-specific language designed by the programmers to describe that system. Functions are the verbs of that language, and classes are the nouns. This is a much older truth. The art of programming is, and has always been, the art of language design.

Master programmers think of systems as stories to be told rather than programs to be written. They use the facilities of their chosen programming language to construct a much richer and more expressive language that can be used to tell that story. Part of that domain-specific language is the hierarchy of functions that describe all the actions that take place within that system. In an artful act of recursion those actions are written to use the very domain-specific language they define to tell their own small part of the story.

To write functions/methods well(and, of course, reviewing other's methods), in addition to following the rules below, you need to let the author know that writing software is like any other kind of writing. When you write a paper or an article,
you get your thoughts down first, then you massage it until it reads well. The first draft might be clumsy and disorganized, so you wordsmith it and restructure it and refine it until it reads the way you want it to read.

When I write functions, they come out long and complicated. They have lots of
indenting and nested loops. They have long argument lists. The names are arbitrary, and there is duplicated code. But I also have a suite of unit tests that cover every one of those clumsy lines of code. So then I massage and refine that code, splitting out functions, changing names, eliminating duplication. I shrink the methods and reorder them. Sometimes I break out whole classes, all the while keeping the tests passing.

In the end, I wind up with functions that follow the rules I’ve laid down in this chapter. I don't write them that way to start. I don’t think anyone could.

\begin{tcolorbox}[breakable, colback=green!10!white, colframe=green!85!black, center title, title = Golden Rule of Functions]
\begin{center}
As small as possible
\end{center}
\end{tcolorbox}

\section{Blocks and Indenting}\index{Functions! Blocks and Indenting}

The blocks within \inlinecode[green]{if} statements, \inlinecode[green]{else} statements, \inlinecode[green]{while} statements, and so on should be one line long. Probably that line should be a function call. Not only does this keep the enclosing function small, but it also adds documentary value because the
function called within the block can have a nicely descriptive name.

This also implies that functions should not be large enough to hold nested structures. Therefore, the indent level of a function should not be greater than one or two. This, of course, makes the functions easier to read and understand.

For example:

\begin{tcolorbox}[breakable, colback=red!10!white, colframe=red!85!black, sidebyside, righthand width = 3cm, tikz lower, label=blocks-and-indenting-bad]

\begin{lstlisting}[language = java, basicstyle=\small]
public static String renderPageWithSetupsAndTeardowns(
        PageData pageData,
        boolean isSuite
) throws Exception {
    boolean isTestPage = pageData.hasAttribute("Test");
    if (isTestPage) {
        WikiPage testPage = pageData.getWikiPage();
        StringBuffer newPageContent = new StringBuffer();
        includeSetupPages(testPage, newPageContent, isSuite);
        newPageContent.append(pageData.getContent());
        includeTeardownPages(testPage, newPageContent, isSuite);
        pageData.setContent(newPageContent.toString());
    }

    return pageData.getHtml();
}
\end{lstlisting}

\tcblower

\path[fill = yellow, draw = yellow!75!red] (0, 0) circle (1cm);
\fill[red] (45:5mm) circle (1mm);
\fill[red] (135:5mm) circle (1mm);
\draw[line width=1mm,red] (230:6mm) arc (145:35:5mm);

\end{tcolorbox}

\begin{tcolorbox}[breakable, colback=green!10!white, colframe=green!85!black, sidebyside, righthand width = 3cm, tikz lower, label=blocks-and-indenting-good]

\begin{lstlisting}[language = java, basicstyle=\small]
public static String renderPageWithSetupsAndTeardowns(
        PageData pageData,
        boolean isSuite
) throws Exception {
    if (isTestPage(pageData)) {
        includeSetupAndTeardownPages(
                pageData,
                isSuite
        );
    }
    return pageData.getHtml();
}
\end{lstlisting}

\tcblower

\path[fill = yellow, draw = yellow!75!red] (0, 0) circle (1cm);
\fill[red] (45:5mm) circle (1mm);
\fill[red] (135:5mm) circle (1mm);
\draw[line width=1mm,red] (215:5mm) arc (215:325:5mm);

\end{tcolorbox}

\begin{marker}
The blocks within \inlinecode[green]{if} statements, \inlinecode[green]{else} statements, \inlinecode[green]{while} statements, and so on should be one line long.
The indent level of a function should not be greater than one or two.
\end{marker}

\section{Do One Thing}\index{Functions!Do One Thing}

While everybody understands the following rule:

\begin{tcolorbox}[breakable, colback=green!10!white, colframe=green!85!black, center title]
\begin{center}
FUNCTIONS SHOULD DO ONE THING. THEY SHOULD DO IT WELL.
THEY SHOULD DO IT ONLY.
\end{center}
\end{tcolorbox}

the problem with this statement is that it is hard to know what “one thing” is. Does Listing~\ref{blocks-and-indenting-good} do one thing? It’s easy to make the case that it’s doing three things:

\begin{enumerate}
    \item Determining whether the page is a test page.
    \item If so, including setups and teardowns.
    \item Rendering the page in HTML.
\end{enumerate}

Notice that the three steps of the function are one level of abstraction below the stated name of the function. We can describe the function by describing it in JavaDoc of this method:

\begin{lstlisting}
"we check to see whether the page is a test page and if so, we include the setups and teardowns. In either case we render the page in HTML."
\end{lstlisting}

If a function does only those steps that are one level below the stated name of the function, then the function is doing one thing. After all, the reason we write functions is to decompose a larger concept (in other words, the name of the function) into a set of steps at the next level of abstraction.

Listing~\ref{blocks-and-indenting-bad} has two levels of abstraction, as proved by our ability to shrink it down. But it would be very hard to meaningfully shrink Listing~\ref{blocks-and-indenting-good}. We could extract the if statement into a function named \inlinecode[green]{includeSetupsAndTeardownsIfTestPage}, but that simply restates the code without changing the level of abstraction.

\subsection{One Level of Abstraction per Function}

In order to make sure our functions are doing “one thing,” we need to make sure that the statements within our function are all at the same level of abstraction.

It is easy to see how a mixing of \inlinecode[yellow]{getHtml()} (very high level abstraction) and \inlinecode[yellow]{string.append("\n")} (very low level abstraction) is violating this rule.

Mixing levels of abstraction within a function is always confusing. Readers may not be able to tell whether a particular expression is an essential concept or a detail. Worse, like broken windows, once details are mixed with essential concepts, more and more details tend to accrete within the function.

\begin{marker}
Do NOT mix level of abstraction in a method
\end{marker}

\section{Reading Code from Top to Bottom: The Stepdown Rule}\index{Functions!Reading Code from Top to Bottom: The Stepdown Rule}

We want the code to read like a top-down narrative.5 We want every function to be followed by those at the next level of abstraction so that we can read the program, descending one level of abstraction at a time as we read down the list of functions.

\begin{marker}
\textbf{The Stepdown Rule} - We want the code to read like a top-down narrative.5 We want every function to be followed by those at the next level of abstraction so that we can read the program, descending one level of abstraction at a time as we read down the list of functions.
\end{marker}

It turns out to be very difficult for programmers to learn to follow this rule and write functions that stay at a single level of abstraction. But learning this trick is also very important. It is the key to keeping functions short and making sure they do "one thing".

\section{Switch Statements}\index{Functions!Switch Statements}

It’s hard to make a small \inlinecode[red]{switch} statement(or an equivalent if-else chains). It’s also hard to make a switch statement
that does one thing. By their nature, switch statements always do N things. Unfortunately we can’t always avoid switch statements, but we can make sure that each \inlinecode[red]{switch} statement is buried in a low-level class and is never repeated. We do this, of course, with
\textbf{polymorphism}.

Consider Listing~\ref{switch-statements-bad}. It shows just one of the operations that might depend on the
type of employee.

\begin{tcolorbox}[breakable, colback=red!10!white, colframe=red!85!black, sidebyside, righthand width = 3cm, tikz lower, label=switch-statements-bad]

\begin{lstlisting}[language = java, basicstyle=\small]
public Money calculatePay(Employee e) {
    switch (e.type) {
        case COMMISSIONED:
            return calculateCommissionedPay(e);
        case HOURLY:
            return calculateHourlyPay(e);
        case SALARIED:
            return calculateSalariedPay(e);
        default:
            // exception
    }
}
\end{lstlisting}

\tcblower

\path[fill = yellow, draw = yellow!75!red] (0, 0) circle (1cm);
\fill[red] (45:5mm) circle (1mm);
\fill[red] (135:5mm) circle (1mm);
\draw[line width=1mm,red] (230:6mm) arc (145:35:5mm);

\end{tcolorbox}

There are several problems with this function:

\begin{enumerate}
    \item it’s large, and when new employee types are added, it will grow
    \item it very clearly does more than one thing
    \item it violates the Single Responsibility Principle because there is more than one reason for it to change
    \item it violates the Open Closed Principle because it must change whenever new types are added
    \item \textbf{WORST} - there are an unlimited number of other functions that will have the same structure. For example we could have \inlinecode[blue]{isPayday(Employee e),} or \inlinecode[blue]{deliverPay(Employee e)} or a host of others. All of which would have the same deleterious structure.
\end{enumerate}

The solution to this problem (see Listing~\ref{switch-statements-good}) is to bury the switch statement in the basement of an ABSTRACT FACTORY and never let anyone see it. The factory will use the
switch statement to create appropriate instances of the derivatives of Employee, and the various functions, such as calculatePay, isPayday, and deliverPay, will be dispatched polymorphically through the Employee interface.

\begin{tcolorbox}[breakable, colback=green!10!white, colframe=green!85!black, sidebyside, righthand width = 3cm, tikz lower, label=switch-statements-good]

\begin{lstlisting}[language = java, basicstyle=\small]
public abstract class Employee {
    public abstract boolean isPayday();
    public abstract Money calculatePay();
    public abstract void deliverPay(Money pay);
}
-----------------
public interface EmployeeFactory {
    public Employee makeEmployee(EmployeeRecord r);
}
-----------------
public class EmployeeFactoryImpl implements EmployeeFactory {

    @Override
    public Employee makeEmployee(EmployeeRecord r) {
        switch (r.type) {
            case COMMISSIONED:
                return new CommissionedEmployee(r) ;
            case HOURLY:
                return new HourlyEmployee(r);
            case SALARIED:
                return new SalariedEmploye(r);
            default:
                // exception
        }
    }
}
\end{lstlisting}

\tcblower

\path[fill = yellow, draw = yellow!75!red] (0, 0) circle (1cm);
\fill[red] (45:5mm) circle (1mm);
\fill[red] (135:5mm) circle (1mm);
\draw[line width=1mm,red] (215:5mm) arc (215:325:5mm);

\end{tcolorbox}

\begin{marker}
Bury the switch statement in the basement of an ABSTRACT FACTORY and never let anyone see it
\end{marker}

The general rule for \inlinecode[yellow]{switch} statements is that they can be tolerated if they

\begin{enumerate}
    \item appear only once
    \item are used to create polymorphic objects
    \item are hidden behind an inheritance relationship so that the rest of the system cannot see them
\end{enumerate}

\begin{marker}
Reduce switch(or if-else chain) statements as much as possible
\end{marker}

\section{Function Arguments}\index{Functions!Function Arguments}

The ideal number of arguments for a function is zero (niladic). Next comes one (monadic), followed closely by two (dyadic). Three arguments (triadic) should be avoided where possible. More than three (polyadic) requires very special justification—and then shouldn't be used anyway

Arguments are hard. They take a lot of conceptual power. Readers
would have had to interpret it each time they saw it. When you are reading the story told by the module, \inlinecode[green]{includeSetupPage()} is easier to understand than \inlinecode[red]{includeSetupPageInto(newPageContent)}. \textbf{The argument is at a different level of abstraction than the function name} and forces you to know a detail that isn't particularly important at that point.

\begin{remark}
Arguments are hard. They take a lot of conceptual power.
\end{remark}

Arguments are also hard from a testing point of view.

Output arguments are harder to understand than input arguments. When we read a function, we are used to the idea of information going in to the function through arguments and out through the return value. We don’t usually expect information to be going out through the arguments. So output arguments often cause us to do a double-take.

\begin{marker}
Reduce the number of arguments as much as possible.

No output argument
\end{marker}

\subsection{Common Monadic Forms}

There are two very common reasons to pass a single argument into a function:

\begin{enumerate}
    \item Asking a question about that argument, as in \inlinecode[green]{boolean fileExists(“MyFile”).}
    \item Operating and transforming it into something else and returning it. For example, transform a file name into content - \inlinecode[green]{InputStream fileOpen(“MyFile”)}
\end{enumerate}

These two uses are what readers expect when they see a function. You should choose names that make the distinction clear, and always use the two forms in a consistent context.

A somewhat less common, but still very useful form for a single argument function, is an \textit{event}. In this form there is an input argument but no output argument. The overall program is meant to interpret the function call as an event and use the argument to alter the state of the system.

\begin{marker}
Use monadic argument appropriately
\end{marker}

Try to avoid any monadic functions that don’t follow these forms. Using an output argument instead of a return value for a transformation is confusing. If a function is going to transform its input argument, the transformation should appear as the return value.

\begin{marker}
Put output argument as return vlaue
\end{marker}

\subsection{Flag Arguments}

Flag arguments are ugly. Passing a boolean into a function is a truly terrible practice. It immediately complicates the signature of the method, loudly proclaiming that this function does more than one thing. It does one thing if the flag is true and another if the flag is false!

\begin{marker}
Split the method into 2 sub-calls to enhance readability 
\end{marker}

\subsection{Dyadic Functions}

There are times, of course, where two arguments are appropriate. For example, it is perfectly reasonable to have \inlinecode[green]{Point p = new Point(0,0);}. Cartesian points naturally take two arguments. However, the two arguments in this case are \textit{ordered components of a single value}! Whereas output-Stream and name, as in \inlinecode[red]{writeField(output-Stream, name)}, have neither a natural cohesion, nor a natural ordering.

Even obvious dyadic functions like \inlinecode[red]{assertEquals(expected, actual)} are problematic.
How many times have you put the \inlinecode[red]{actual} where the \inlinecode[red]{expected} should be? The two arguments have no natural ordering. The expected, actual ordering is a convention that requires practice to learn. This is one of the reasons I do not use junit for testing.

Dyads aren’t evil, and you will certainly have to write them. However, you should be aware that they come at a cost and should take advantage of what mechanims may be available to you to convert them into monads. For example, you might make the \inlinecode[green]{writeField} method a member of \inlinecode[green]{outputStream} so that you can say \inlinecode[green]{outputStream. writeField(name)}. Or you might make the \inlinecode[green]{outputStream} a member variable of the current class so that you don’t have to pass it. Or you might extract a new class like \inlinecode[green]{FieldWriter} that takes the \inlinecode[green]{outputStream} in its constructor and has a write method.

\begin{marker}
Design natural-ordering or natural-cohesion bi-arguments
\end{marker}

\section{Argument Objects}\index{Functions!Argument Objects}

When a function seems to need more than two or three arguments, it is likely that some of those arguments ought to be wrapped into a class of their own.

Consider, for example, the difference between the two following declarations:

\begin{tcolorbox}[breakable, colback=green!10!white, colframe=green!85!black, center title, sidebyside]

\begin{lstlisting}[language = java, basicstyle=\small]
Circle makeCircle(double x, double y, double radius);
\end{lstlisting}

\tcblower

\begin{lstlisting}[language = java, basicstyle=\small]
Circle makeCircle(Point center, double radius);
\end{lstlisting}

\end{tcolorbox}

Reducing the number of arguments by creating objects out of them may seem like cheating, but it’s not. When groups of variables are passed together, the way x and y are in the example above, they are likely part of a concept that deserves a name of its own.

\section{Argument Lists}\index{Functions!Argument Lists}

Sometimes we want to pass a variable number of arguments into a function. Consider, for example:

\begin{tcolorbox}[breakable, colback=green!10!white, colframe=green!85!black]
\begin{lstlisting}[language = java, basicstyle=\small]
String.format("%s worked %.2f hours.", name, hours);
\end{lstlisting}
\end{tcolorbox}

If the variable arguments are all treated identically, as they are in the example above, then they are equivalent to a single argument of type List. By that reasoning, this method is actually dyadic. Indeed, the declaration of String.format as shown below is clearly dyadic:

\begin{tcolorbox}[breakable, colback=green!10!white, colframe=green!85!black]
\begin{lstlisting}[language = java, basicstyle=\small]
public String format(String format, Object... args)
\end{lstlisting}
\end{tcolorbox}

So all the same rules apply. Functions that take variable arguments can be monads, dyads, or even triads. But it would be a mistake to give them more arguments than that.

\begin{tcolorbox}[breakable, colback=green!10!white, colframe=green!85!black, sidebyside, righthand width = 3cm, tikz lower]

\begin{lstlisting}[language = java, basicstyle=\small]
void monad(Integer... args);
void dyad(String name, Integer... args);
void triad(String name, int count, Integer... args);
\end{lstlisting}

\tcblower

\path[fill = yellow, draw = yellow!75!red] (0, 0) circle (1cm);
\fill[red] (45:5mm) circle (1mm);
\fill[red] (135:5mm) circle (1mm);
\draw[line width=1mm,red] (215:5mm) arc (215:325:5mm);

\end{tcolorbox}

\section{Coherent method name and arguments}\index{Functions!Coherent method name and arguments}

Choosing good names for a function can go a long way toward explaining the intent of the function and the order and intent of the arguments. In the case of a monad, the function and argument should form a very nice verb/noun pair. For example, \inlinecode[green]{write(name)} is very evocative. Whatever this “name” thing is, it is being “written.”

\section{Have No Side Effects}\index{Functions!Have No Side Effects}

\subsection{What is Side Effects}

Side effects are lies. Your function promises to do one thing, but it also does other hidden things. Sometimes it will make unexpected changes to the variables of its own class. Sometimes it will make them to the parameters passed into the function or to system globals. In either case they are devious and damaging mistruths that often result in strange temporal couplings and order dependencies.

\subsection{Example}

\begin{tcolorbox}[breakable, colback=red!10!white, colframe=red!85!black, sidebyside, righthand width = 3cm, tikz lower]

\begin{lstlisting}[language = java, basicstyle=\small]
public class UserValidator {
    
    private Cryptographer cryptographer;

    public boolean checkPassword(String userName, String password) {
        User user = UserGateway.findByName(userName);
        if (user != User.NULL) {
            String codedPhrase = user.getPhraseEncodedByPassword();
            String phrase = cryptographer.decrypt(codedPhrase, password);
            if ("Valid Password".equals(phrase)) {
                Session.initialize();
                return true;
            }
        }
        return false;
    }
}
\end{lstlisting}

\tcblower

\path[fill = yellow, draw = yellow!75!red] (0, 0) circle (1cm);
\fill[red] (45:5mm) circle (1mm);
\fill[red] (135:5mm) circle (1mm);
\draw[line width=1mm,red] (230:6mm) arc (145:35:5mm);

\end{tcolorbox}

The side effect is the call to \inlinecode[red]{Session.initialize()}, of course. The \inlinecode[red]{checkPassword} function, by its name, says that it checks the password. The name does not imply that it initializes the session. So a caller who believes what the name of the function says runs the risk of erasing the existing session data when he or she decides to check the validity of the user.

This side effect creates a temporal coupling. That is, \inlinecode[red]{checkPassword} can only be called at certain times (in other words, when it is safe to initialize the session). If it is called out of order, session data may be inadvertently lost. Temporal couplings are confusing, especially when hidden as a side effect. If you must have a temporal coupling, you should make it clear in the name of the function. In this case we might rename the function \inlinecode[red]{checkPasswordAndInitializeSession}, though that certainly violates "Do one thing."

\begin{marker}
Watch out for side effects in review - does the method do more than what method name implies?
\end{marker}

\subsection{Output Arguments}

Output argument makes it hard to read. For example:

\begin{tcolorbox}[breakable, colback=red!10!white, colframe=red!85!black]
appendFooter(s);
\end{tcolorbox}

Does this function append s as the footer to something? Or does it append some footer to s? Is s an input or an output? It doesn’t take long to look at the function signature and see:

\begin{tcolorbox}[breakable, colback=red!10!white, colframe=red!85!black]
public void appendFooter(StringBuffer report)
\end{tcolorbox}

This clarifies the issue, but only at the expense of checking the declaration of the function. Anything that forces you to check the function signature is equivalent to a double-take. It’s a cognitive break and should be avoided.

\begin{marker}
Anything that forces you to check the function signature is equivalent to a double-take which should be avoided
\end{marker}

In general output arguments should be avoided. If your function must change the state of something, \textit{have it change the state of its owning objec}

\section{Command Query Separation}\index{Functions!Command Query Separation}

Functions should either do something or answer something, but not both. Either your
function should change the state of an object, or it should return some information about that object. Doing both often leads to confusion. Consider, for example, the following function:

\begin{tcolorbox}[breakable, colback=red!10!white, colframe=red!85!black, sidebyside, righthand width = 3cm, tikz lower]
\begin{lstlisting}[language = java, basicstyle=\small]
public boolean set(String attribute, String value);
\end{lstlisting}

\tcblower

\path[fill = yellow, draw = yellow!75!red] (0, 0) circle (1cm);
\fill[red] (45:5mm) circle (1mm);
\fill[red] (135:5mm) circle (1mm);
\draw[line width=1mm,red] (230:6mm) arc (145:35:5mm);
\end{tcolorbox}

This function sets the value of a named attribute and returns true if it is successful and false if no such attribute exists. This leads to odd statements like this:

\begin{tcolorbox}[breakable, colback=red!10!white, colframe=red!85!black, sidebyside, righthand width = 3cm, tikz lower]
\begin{lstlisting}[language = java, basicstyle=\small]
if (set("username", "unclebob"))...
\end{lstlisting}

\tcblower

\path[fill = yellow, draw = yellow!75!red] (0, 0) circle (1cm);
\fill[red] (45:5mm) circle (1mm);
\fill[red] (135:5mm) circle (1mm);
\draw[line width=1mm,red] (230:6mm) arc (145:35:5mm);
\end{tcolorbox}

What does it mean? It is hard to answer without reading the source code of the method(or its doc). The real solution is to separate the command from the query so that the ambiguity cannot occur:

\begin{tcolorbox}[breakable, colback=green!10!white, colframe=green!85!black, sidebyside, righthand width = 3cm, tikz lower]

\begin{lstlisting}[language = java, basicstyle=\small]
if (attributeExists("username")) {
    setAttribute("username", "unclebob");
    ...
}
\end{lstlisting}

\tcblower

\path[fill = yellow, draw = yellow!75!red] (0, 0) circle (1cm);
\fill[red] (45:5mm) circle (1mm);
\fill[red] (135:5mm) circle (1mm);
\draw[line width=1mm,red] (215:5mm) arc (215:325:5mm);

\end{tcolorbox}

\section{Prefer Exceptions to Returning Error Codes}\index{Functions~Prefer Exceptions to Returning Error Codes}

Returning error codes from command functions is a subtle violation of command query
separation:

\begin{tcolorbox}[breakable, colback=red!10!white, colframe=red!85!black, sidebyside, righthand width = 3cm, tikz lower, label=exception-over-err-code-bad]
\begin{lstlisting}[language = java, basicstyle=\small]
if (deletePage(page) == E_OK) {
    if (registry.deleteReference(page.name) == E_OK) {
        if (configKeys.deleteKey(page.name.makeKey()) == E_OK){
            logger.log("page deleted");
        } else {
            logger.log("configKey not deleted");
        }
    } else {
        logger.log("deleteReference from registry failed");
    }
} else {
    logger.log("delete failed");
    return E_ERROR;
}
\end{lstlisting}

\tcblower

\path[fill = yellow, draw = yellow!75!red] (0, 0) circle (1cm);
\fill[red] (45:5mm) circle (1mm);
\fill[red] (135:5mm) circle (1mm);
\draw[line width=1mm,red] (230:6mm) arc (145:35:5mm);
\end{tcolorbox}

Note the first line in Listing~\ref{exception-over-err-code-bad}. When you return an error code, you create the problem that the caller must \textit{deal with the error immediately}.

On the other hand, if you use exceptions instead of returned error codes, then the error processing code can be separated from the happy path code and can be simplified:

\begin{tcolorbox}[breakable, colback=green!10!white, colframe=green!85!black, sidebyside, righthand width = 3cm, tikz lower]

\begin{lstlisting}[language = java, basicstyle=\small]
try {
    deletePage(page);
    registry.deleteReference(page.name);
    configKeys.deleteKey(page.name.makeKey());
} catch (Exception e) {
    logger.log(e.getMessage());
}
\end{lstlisting}

\tcblower

\path[fill = yellow, draw = yellow!75!red] (0, 0) circle (1cm);
\fill[red] (45:5mm) circle (1mm);
\fill[red] (135:5mm) circle (1mm);
\draw[line width=1mm,red] (215:5mm) arc (215:325:5mm);

\end{tcolorbox}

Another advantage of using Exceptions is that returning error codes usually implies that there is some class or enum in which all the error codes are defined.

\begin{tcolorbox}[breakable, colback=red!10!white, colframe=red!85!black, sidebyside, righthand width = 3cm, tikz lower, label=exception-over-err-code-bad]
\begin{lstlisting}[language = java, basicstyle=\small]
public enum Error {
    OK,
    INVALID,
    NO_SUCH,
    LOCKED,
    OUT_OF_RESOURCES,
    WAITING_FOR_EVENT;
}
\end{lstlisting}

\tcblower

\path[fill = yellow, draw = yellow!75!red] (0, 0) circle (1cm);
\fill[red] (45:5mm) circle (1mm);
\fill[red] (135:5mm) circle (1mm);
\draw[line width=1mm,red] (230:6mm) arc (145:35:5mm);
\end{tcolorbox}

Classes like this are a dependency magnet; many other classes must import and use
them. Thus, when the Error enum changes, all those other classes need to be recompiled and redeployed.11 This puts a negative pressure on the Error class. Programmers don't want to add new errors because then they have to rebuild and redeploy everything. So they reuse old error codes instead of adding new ones.

When you use exceptions rather than error codes, then new exceptions are derivatives of the exception class. They can be added without forcing any recompilation or redeployment

\begin{marker}
Exception-handling, instead of Error Code, reduces coupling
\end{marker}

\section{Extract Try/Catch Blocks}\index{Functions!Extract Try/Catch Blocks}

Try/catch blocks are ugly in their own right. They confuse the structure of the code and mix error processing with normal processing. So it is better to extract the bodies of the try and catch blocks out into functions of their own.

Functions should do one thing. Error handing is one thing. Thus, a function that handles errors should do nothing else. This implies (as in the example above) that if the keyword try exists in a function, it should be the very first word in the function and that there should be nothing after the catch/finally blocks.

\begin{marker}
Put try-catch in a method and nothing else
\end{marker}
